\documentclass{article}
\author{Kyarigwo}
\usepackage[margin=0.6in]{geometry}
\usepackage{bm}
\usepackage{amsmath}
\usepackage{amssymb}
\usepackage{parskip}
\title{Risc-v summary}
\begin{document}

\begin{center}

  \textbf{Registers}

  \begin{tabular}{|c|c|c|c||c|c|c|c|}
    \hline
    name & othername & description & save? &  name & othername & description & save? \\
    \hline
    x0 & zero & always 0 & yes? & x16 & a6  & argument 6 & - \\
    x1 & ra  & return address  & - & x17 & a7  & argument 7  & - \\
    x2 & sp  & stack pointer  & yes & x18 & s2  & saved  & yes \\
    x3 & gp  & global pointer  & - & x19 & s3  & saved  & yes \\
    x4 & tp  & thread pointer  & - & x20 & s4  & saved  & yes \\
    x5 & t0  & temp  & - & x21 & s5  & saved  & yes \\
    x6 & t1  & temp  & - & x22 & s6  & saved  & yes \\
    x7 & t2  & temp  & - & x23 & s7  & saved  & yes \\
    x8 & s0, fp & saved/frame pointer & yes & x24 & s8  & saved  & yes \\
    x9 & s1  & saved  & yes & x25 & s9  & saved  & yes \\
    x10 & a0  & argument 0  & - & x26 & s10  & saved  & yes \\
    x11 & a1  & argument 1  & - & x27 & s11  & saved  & yes \\
    x12 & a2  & argument 2  & - & x28 & t3  & temp  & - \\
    x13 & a3  & argument 3  & - & x29 & t4  & temp  & - \\
    x14 & a4  & argument 4  & - & x30 & t5  & temp  & - \\
    x15 & a5  & argument 5  & - & x31 & t6  & temp  & - \\
  \hline
  \end{tabular}


  \textbf{Addition and subtraction}

  \begin{tabular}{|c|c|c|c|}
    \hline
    name          & format             & example          & definition    \\
    \hline
    add immediate & addi rd, r1, immed & addi x4, x9, 123 & x4 = x9 + 123 \\
    add           & add  rd, r1, r2    & add x4, x9, x13  & x4 = x9 + x13 \\
    subtract      & sub  rd, r1, r2    & sub x4, x9, x13  & x4 = x9 - x13 \\
    *negate        & neg rd, r2         & neg x4, x9       & x4 = -x9      \\
    \hline
  \end{tabular}

  \textbf{Multiplication and division}

  \begin{tabular}{|c|c|c|c|}
    \hline
    name          & format             & example          & definition    \\
    \hline
    multiply & mul rd, r1, r2 & mul x4, x9, x13 & x4 = x9 * x13 \\
    multiply high bits (signed) & mulh rd, r1, r2 & mulh x4, x9, x13 & x4 = highbits(x9 * x13) \\
    multiply high bits (unsigned) & mulhu rd, r1, r2 & mulhu x4, x9, x13 & x4 = highbits(x9 * x13) \\
    multiply high bits (signed and unsigned) & mulhsu rd, r1, r2 & mulhsu x4, x9, x13 & x4 = highbits(x9 * x13) \\
    divide (signed) & div rd, r1, r2 & div x4, x9, x13 & x4 = x9 div x13 \\
    divide (unsigned) & divu rd, r1, r2 & divu x4, x9, x13 & x4 = x9 div x13 \\
    remainder (signed) & rem rd, r1, r2 & rem x4, x9, x13 & x4 = x9 rem x13 \\
    remainder (unsigned) & remu rd, r1, r2 & remu x4, x9, x13 & x4 = x9 rem x13 \\
    \hline
  \end{tabular}

  \textbf{Loads}

  \begin{tabular}{|c|c|c|c|}
    \hline
    name            & format            & example          & definition \\
    \hline
    load byte (signed) & lb rd, immed(r1) & lb x4, 1234(x9) & x4 = mem[x9 + 1234] \\
    load byte (unsigned) & lbu rd, immed(r1) & lbu x4, 1234(x9) & x4 = mem[x9 +
    1234] \\
    *load byte & lb rd, immed & lb x4, var & x4 = mem[var] \\
    load halfword (signed) & lh rd, immed(r1) & lh x4, 1234(x9) & x4 = Mem[x9+1234] \\
    load halfword (unsigned) & lhu rd, immed(r1) & lhu x4, 1234(x9) & x4 =
    Mem[x9+1234] \\
    *load halfword & lh rd, immed & lh x4, var & x4 = mem[var] \\
    load word (signed)  & lw rd, immed(r1) & lw x4, 1234(x9) & x4 = Mem[x9+1234] \\
    load word (unsigned)  & lwu rd, immed(r1) & lwu x4, 1234(x9) & x4 =
    Mem[x9+1234] \\
    *load word & lw rd, immed & lw x4, var & x4 = mem[var] \\
    *load immediate                   & li rd, immed       & li x4, 123        & x4 = 123 \\
    \hline
  \end{tabular}

  \textbf{Stores}

  \begin{tabular}{|c|c|c|c|}
    \hline
    name            & format            & example          & definition \\
    \hline
    store byte & sb rd, immed(r1) & sb x4, 1234(x9) & mem[x9 + 1234] = x4 \\
    *store byte & sb rd, immed, r1 & sb x4, var, a0 & mem[var] = x4, a0 is temp\\
    store halfword & sh rd, immed(r1) & sh x4, 1234(x9) & mem[x9 + 1234] = x4 \\
    *store halfword & sh rd, immed, r1 & sh x4, var, a0 7 mem[var] = x4, a0 is temp\\
    store word & sw rd, immed(r1) & sw x4, 1234(x9) & mem[x9 + 1234] = x4 \\
    *store word & sw rd, immed, r1 & sw x4, var, a0 & mem[var] = x4, a0 is temp\\
    \hline
  \end{tabular}

  \textbf{misc}

  \begin{tabular}{|c|c|c|c|}
    \hline
    name                             & format             & example           & definition      \\
    \hline
    *nop                              & nop                & nop               & does nothing    \\
    *move register to register        & mv rd, r1          & mv x4, x9         & x4 = x9         \\
    load upper immediate             & lui rd, immed      & lui x4 0x12345    & x4 = 0x12345000
    \\
    add upper immediate to pc        & auipc rd, immed    & auipc x4, 0x12345 & x4 = pc +
    (0x12345$<<$12)                                                                             \\
    *load address                     & la rd, address     & la x4, loop       & x4 = loop
    \\
    set if less then (signed)        & slt rd, r1, r2     & slt x4, x9, x13   & x4 = (x9 $<$
    x13) ? 1 : 0                                                                                \\
    set less then immediate (signed) & slti rd, r1, immed & slti x4, x9, 123  &
    x4 = (x9 $<$ 123) ? 1 : 0                                                                   \\
    \hline
  \end{tabular}

\textbf{Jumps}

  \begin{tabular}{|c|c|c|c|}
    \hline
    name            & format            & example          & definition \\
    \hline
    *jump and link & jal rd, immed & jal x1, loop & goto loop, x1 = pc \\
    *jump & j immed & jal loop & goto loop \\
    *jump and link register & jalr rd, r1, immed & jalr x1, x4, loop & goto loop
    + x4, x1 = pc \\
    *jump register & jr r1 & jr x4 & goto r1\\
    *return & ret & ret & goto x1 \\
    *call far & call immed & call func & goto func, x1 = pc \\
    *tail call far & tail immed & tail func & goto func, discard pc\\
    \hline
  \end{tabular}

  jalr reg is the same as jalr x0, reg, 0.


  \textbf{Branches}

  \begin{tabular}{|c|c|c|c|}
    \hline
    name                                       & format                & example           & definition            \\
    \hline
    branch if equal                            & beq r1, r2, immed     & beq x4, x9, loop  & if x4 == x9
    goto loop                                                                                                      \\
    branch if not equal                        & bne r1, r2, immed     & bne x4, x9, loop  & if x4 != x9 goto loop \\
    branch if less than (signed)               & blt r1, r2, immed     & blt x4, x9, loop  & if x4 < x9 goto loop  \\
    *branch if less than or equal (signed)      & ble r1, r2, immed     & ble x4, x9, loop  & if x4 <= x9 goto loop \\
    *branch if greater then (signed)            & bgt r1, r2, immed     & bgt x4, x9, loop  & if x4 > x9 goto loop  \\
    branch if greater than or equal (signed)   & bge r1, r2, immed     & bge x4, x9,
    loop                                       & if x4 >= x9 goto loop                                             \\
    branch if less than (unsigned)             & bltu r1, r2, immed    & bltu x4, x9, loop & if x4 < x9 goto loop  \\
    *branch if less than or equal (unsigned)    & bleu r1, r2, immed    & bleu x4, x9, loop & if x4 <= x9 goto loop \\
    *branch if greater then (unsigned)          & bgtu r1, r2, immed    & bgtu x4, x9, loop & if x4 > x9 goto loop  \\
    branch if greater than or equal (unsigned) & bgeu r1, r2, immed    & bgeu x4, x9, loop & if x4 >= x9 goto loop \\
    \hline
  \end{tabular}

  If r2 is omitted, it is assumed to be x0.

    \textbf{Bitwise}

  \begin{tabular}{|c|c|c|c|}
    \hline
    name          & format             & example          & definition        \\
    \hline
    and immediate & andi rd, r1, immed & andi x4, x9, 123 & x4 = x9 \& 123    \\
    and           & and rd, r1, r2     & and x4, x9, x13  & x4 = x9 \& x13    \\
    or immediate  & ori rd, r1, immed  & ori x4, x9, 123  & x4 = x9 \(|\) 123 \\
    or            & or rd, r1, r2      & or x4, x9, x13   & x4 = x9 \(|\) x13 \\
    xor immediate & xori rd, r1, immed & xori x4, x9, 123 & x4 = x9 xor 123   \\
    xor           & xor rd, r1, r2     & xor x4, x9, x13  & x4 = x9 xor x13   \\
    *not   & not rd, r1         & not x4, x9       & x4 = !x9          \\
    \hline
  \end{tabular}



  \textbf{Shifts}

  \begin{tabular}{|c|c|c|c|}
    \hline
    name                             & format             & example          & definition       \\
    \hline
    shift left logical immediate     & slli rd, r1, immed & slli x4, x9, 5   & x4 = x9
    $<<$ 5                                                                                      \\
    shift left logical               & sll rd, r1, r2     & sll x4, x9, x13  & x4 = x9 $<<$ x13 \\
    shift right logical immediate    & srli rd, r1, immed & srli x4, x9, 5   & x4 =
    x9 $>>$ 5                                                                                   \\
    shift right logical              & srl rd, r1, r2     & srl x4, x9, x13  & x4 = x9 $>>$ x13 \\
    shift right Arithmetic immediate & srai rd, r1, immed & srai x4, x9, 5   & x4
    = x9
    $>>$ 5 sign extended                                                                        \\
    shift right Arithmetic           & sra rd, r1, r2     & srai x4, x9, x13 & x4
    = x9 $>>$ x13 sign extended                                                                 \\
    \hline
  \end{tabular}

  \textbf{Set conditions}

  \begin{tabular}{|c|c|c|c|}
    \hline
    name                               & format              & example           & definition            \\
    \hline
    set if less then (signed)          & slt rd, r1, r2      & slt x4, x9, x13   & x4 = (x9 $<$
    x13) ? 1 : 0                                                                                         \\
    set less then immediate (signed)   & slti rd, r1, immed  & slti x4, x9, 123  &
    x4 = (x9 $<$ 123) ? 1 : 0
    \\
    set if greater than (signed)       & sgt rd, r1, r2      & sgt x4, x9, x13   & x4 = (x9
    $>$ x13 ) ? 1 : 0
    \\
    set if less then (unsigned)        & sltu rd, r1, r2     & slty x4, x9, x13  & x4 = (x9 $<$
    x13) ? 1 : 0                                                                                         \\
    set less then immediate (unsigned) & sltiu rd, r1, immed & sltiu x4, x9, 123 &
    x4 = (x9 $<$ 123) ? 1 : 0                                                                            \\
    set if greater than (unsigned)     & sgtu rd, r1, r2     & sgtu x4, x9, x13  & x4 = (x9
    $>$ x13 ) ? 1 : 0
    \\
    *set if equal to zero               & seqz rd, r1         & seqz x4, x9       & x4 = (x9==0) ? 1 : 0  \\
    *set if not equal to zero           & snez rd, r1         & snez x4, x9       & x4 = (x9!= 0) ? 1 : 0
    \\
    *set if less then zero              & sltz rd, r1         & sltz x4, x9       & x4 = (x9<0) ? 1 : 0   \\
    *set if greater then zero           & sgtz rd, r1         & sgtz x4, x9       & x4 = (x9>0) ? 1 :  0  \\
    \hline
  \end{tabular}
\end{center}

\end{document}